\documentclass[11pt, a4paper]{article}
\usepackage[utf8]{inputenc}

\usepackage[ngerman]{babel}

% basic page layout
\usepackage[margin=1in, left=1cm, right=1cm]{geometry}

\usepackage{fancyhdr}
\pagestyle{fancy}
\fancyhf{}
\fancyhead[L]{\sc \LaTeX Cheat Sheet}
\fancyfoot[C]{\thepage}
\fancyfoot[R]{G. Wellecke}

\renewcommand{\headrulewidth}{1pt}
\renewcommand{\footrulewidth}{1pt}

\setlength\parindent{0pt}

% writing vectors as \vv*{vector}{index} or \vv{vector} respectively
\usepackage[d]{esvect}

% for writing math
\usepackage{amsmath}
\usepackage{amssymb}
\usepackage{amsfonts}
\usepackage{amsthm}

% for changing enumeration markers
% \begin{enumerate}[label=(\roman*)]
\usepackage{enumitem}

% for pretty quotes. use: \textquote{}
\usepackage{csquotes}
\usepackage[backend = biber, sorting = none]{biblatex}
\bibliography{sources}

%SI EINHEITEN
\usepackage[mode=text,decimalsymbol=comma,exponent-product=\cdot,separate-uncertainty = true]{siunitx}
%SI-Einheiten über \SI{<num>}{<unit>}, \ang{;;} Winkel in Grad/Min/Sek, \si{<unit>}
\sisetup{locale=DE}

%Tabellensatz
\usepackage{booktabs}
    %Bsp
    %\toprule wird zu Beginn der Tabelle gesetzt oder \cline{2-3} für Linie von Zeile 2 bis 3
    %\midrule bzw. \cmidrule werden innerhalb der Tabelle als horizontale Trennstriche verwendet
    %\bottomrule setzt den Schlussstrich unter die Tabelle.

% Zur Graphikausgabe
%Beipiel: \includegraphics[width=\textwidth]{grafik.png}
\usepackage{graphicx}

% Mehrere Abbildungen nebeneinander
% Beispiel:
% \begin{figure}[htb]
%   \centering
%   \subfigure[Beschriftung 1\label{fig:label1}]
%   {\includegraphics[width=0.49\textwidth]{grafik1}}
%   \hfill
%   \subfigure[Beschriftung 2\label{fig:label2}]
%   {\includegraphics[width=0.49\textwidth]{grafik2}}
%   \caption{Beschriftung allgemein}
%   \label{fig:label-gesamt}
% \end{figure}
\usepackage{subfigure}


% Für Links
\usepackage{hyperref}
    % Bsp: \url{www.wikipedia.de}

%Tabellen an Stelle zwingen mit [H]
\usepackage{float}

% %code einbinden
\usepackage{minted}
% \usemintedstyle{borland}
%     %Bsp: \begin{minted}{python}\end{minted}
\usepackage{xcolor}

%\begin{longtable}\end{longtable} Tabellen über mehrere Seiten
\usepackage{longtable}

\usepackage[]{todonotes}


\begin{document}

	{\bf \sc \Large Protokoll Cheat Sheet}

	\paragraph{Grundlegender \textquote{Stil} eines Protokolls:}
	\begin{enumerate}[label=(\roman*)]
		\item Pr"asens
		\item Rechtschreibung (Spellcheck, Korrektur lesen, $\ldots$)
        \item \underline{kein:} ich , wir, man (d.\,h. objektiv schreiben)
		\item Die Auswertung muss Formeln f"ur die bestimmten Werte enthalten (oder Verweise darauf) und auch die \underline{Fehlerformeln}!
		\item Generell unterschriebenes \underline{Messprotokoll} mit abgeben
		\item Durchf"uhrung nicht wie in Praktikumsanleitung, sondern als Beschreibung, die auch ohne vor dem spezifischen Versuchsaufbau zu sitzen verstanden werden kann. S"atze wie \textquote{lege Schalter 3 um} vermeiden und lieber schreiben \textquote{Schalte jetzt den zus"atzlichen Widerstand $R_3$ zu}.\\ Das Experiment sollte danach von einem anderen Physiker reproduziert werden k"onnen.
	\end{enumerate}

	\paragraph{Bib\TeX{} f"ur Literaturlisten:}
	In den header der \LaTeX{} Datei:
	\begin{minted}{tex}
\usepackage{csquotes}
\usepackage[sortcites, backend = biber, sorting = none]{biblatex}
\bibliography{sources}
	\end{minted}
	Hier k"onnte man noch eine Sortierung definieren (siehe Google). Unter \emph{sources} muss der Dateipfad zur jeweiligen Bib\TeX{} Datei angegeben werden.\\
    Eine Bib\TeX{} Datei hat folgenden Syntax (man beachte die Kommasetzung):
    \inputminted{tex}{sources.bib}

	\paragraph{Zitieren und Referenzieren:}
	Mit obigem Bib\TeX{} l"asst sich dann mit
	\begin{minted}{tex}
\cite[S.90]{exphy1_dem}
	\end{minted}
	oder mit
	\begin{minted}{tex}
\cite{lp}
	\end{minted}
	im Text zitiert werden. Obige Zitate s"ahen dann so aus: Demtr"oder, \cite[S. 90]{exphy1_dem} oder das LP, \cite{lp}. Weitere Info zur Syntax auf \url{https://en.wikipedia.org/wiki/BibTeX}.\\
	Mit dem Befehl
	\begin{minted}{tex}
\printbibliography[title = {Literatur}, heading=bibintoc]
	\end{minted}
    l"asst sich die Literaturliste dann am Ende der Datei drucken. Wichtig: obiger Befehl muss noch vor \emph{end(document)} stehen. Mit \texttt{heading=bibintoc} wird die Literaturliste automatisch in das Inhaltverzeichnis eingetragen.\\

	Au\ss erdem lassen sich Formeln, Abbildungen, Tabellen, etc. mit einem \emph{label} versehen um so sp"ater darauf zu verweisen. Wichtig ist, dass das \emph{label} innerhalb der Umgebung ist, auf die verwiesen werden soll.

	\begin{minted}{tex}
\begin{Umgebung fuer Tabelle}
...
\label{tab: bsp_table}
\end{Umgebung fuer tabelle}
...
Siehe Abb. \ref{fig: Bild1}.
	\end{minted}

	\paragraph{Sch"one Tabellen:}
	In den header:
	\begin{minted}{tex}
\usepackage{booktabs}
	\end{minted}
	Und dann Tabellen mit:
	\begin{minted}{tex}
\begin{table}[h]
    \centering
    \caption{Eine Beispieltabelle.}
    \begin{tabular}{cc}
        \\\toprule
        Titel links & Titel rechts \\
        \midrule
        hier steht & dann ganz, ganz, ganz \\
        viel & toller Inhalt! \\
        \bottomrule
    \end{tabular}
    \label{tab: bsp_table}
\end{table}
	\end{minted}
	Im Idealfall enthalten Tabellen nur horizontale, aber keine vertikalen Linien.\\
	Das {\tt [h]} \textquote{zwingt} die Tabelle an diese Stelle (h = here) im \LaTeX{} Dokument und ist Teil des Packages \emph{float}.\\
	Das sieht dann so aus:
	\begin{table}[h]
	    \centering
	    \caption{Eine Beispieltabelle.}
	    \begin{tabular}{cc}
	        \\\toprule Titel links & Titel rechts \\
	        \midrule
	        hier steht & dann ganz, ganz, ganz \\
	        viel & toller Inhalt! \\
	        \bottomrule
	    \end{tabular}
	    \label{tab: bsp_table}
	\end{table}

	Beachte die Nutzung von \emph{toprule, midrule \& bottomrule}. Sowie bereits erw"ahnte Nutzung von labels (siehe oben). Au\ss erdem kann eine \emph{Caption} definiert werden. Die Nummerierung in der Tabelle macht \LaTeX{} dann selber, siehe Tab. \ref{tab: bsp_table}.

	Um in der pdf Datei Verlinkungen zu erzeugen und URLs, wie oben, zu drucken, l"asst sich noch das Package
	\begin{minted}{tex}
\usepackage{hyperref}
	\end{minted}
	einbinden.

	\paragraph{Dynamische Klammern in Mathe-Umgebungen:}
	\begin{minted}[breaklines=true]{tex}
\begin{equation} \label{eq: bsp_gleichung}
    f(x) = \left(\frac{\sum_{n=0}^\infty \left(\frac{1}{n!}\right)}{f'(x)}\right).
\end{equation}
	\end{minted}
    Mit den Befehlen \emph{left} und \emph{right} lassen sich alle Klammern dynamisch auf die richtige L"ange setzen (funktioniert auch mit anderen Klammern, einfach mal ausprobieren!). Gl. \eqref{eq: bsp_gleichung} sieht dann so aus:
	\begin{equation} \label{eq: bsp_gleichung}
		f(x) = \left(\frac{\sum_{n=0}^\infty \left(\frac{1}{n!}\right)}{f'(x)}\right).
	\end{equation}
	Beachte auch hier die Nutzung des labels.

	\paragraph{Weitere interessante Packages:}
	Hier gilt es zu googlen, wenn ihr es mal braucht.
	\begin{minted}[breaklines=true]{tex}
% zum Setup der Seitenränder
\usepackage{geometry}

% schöne Header und Footer
\usepackage{fancyhdr}

% zur Vervollständigung der Matheumgebungen
\usepackage{amsmath}
\usepackage{amssymb}
\usepackage{amsfonts}
\usepackage{amsthm}

% kann sehr nuetzlich sein
\usepackage{mathtools}
\usepackage{tensor}
\usepackage{hhtensor}

% andere Enumerations (siehe ganz oben - Grundlegender Stil)
\usepackage{enumitem}

% zum Einfügen von Code in LaTeX, wie ich es hier z.B. getan habe
% Achtung beim kompilieren mit latexmk: nutze `-shell-escape` Option
\usepackage{minted}

% oben erwähnt
\usepackage{float}

% SI EINHEITEN
% mögliche Optionen hierfür: [mode=text,decimalsymbol=comma,
% exponent-product=\cdot,separate-uncertainty = true]
\usepackage{siunitx}
% SI-Einheiten über \SI{<num>}{<unit>}, \ang{;;} Winkel in Grad/Min/Sek, \si{<unit>}
\sisetup{locale=DE}

% nuetzlich, wenn ihr laengere Arbeiten schreibt
\usepackage{todonotes}

% wenn ihr euer Projekt gern in mehreren, einzeln kompilierbaren Dateien strukturieren wollt
\usepackage{subfiles}
	\end{minted}

	\paragraph{Plots und Fits:}
	\begin{enumerate}[label=(\Roman*)]
	 	\item auf Achsenbeschriftung (mit Einheiten) achten
	 	\item keine Datenpunkte oder Funktionen sollten die Legende "uberschneiden
	 	\item grid aktivieren, um Ablesen der Daten zu vereinfachen
        \item Stellt euch die Frage: Ist der Plot klar oder enth"alt er unn"otige Information? Machen die ticks und deren Beschriftung Sinn?
        \item bei Fits immer die Methode, die gefittete Funktion und das Ergebnis angeben (h"aufig ist eine Tabelle eine gute Wahl)
	 	\item die G"ute des Fits diskutieren, z.\,B. anhand von $\chi^2/\mathrm{ndf}$
	 \end{enumerate}

     \paragraph{Compilation:} Wie nun aber daraus eine PDF-Datei machen? Wenn ihr \emph{Sharelatex / Overleaf} vermeiden wollt, dann kann ich euch einen guten Editor / IDE nahe legen (z.\,B. \texttt{vim} mit Plugin \texttt{vimtex}), der euch dann die Arbeit abnimmt. Wer es ganz klassisch m"ochte kann auch direkt in der command line kompilieren, daf"ur nimmt man am besten \texttt{latexmk} (siehe \url{https://mg.readthedocs.io/latexmk.html}).

	 \printbibliography[title = {Literatur}]
\end{document}
